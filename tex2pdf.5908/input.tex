\documentclass[12pt,]{article}
\usepackage{lmodern}
\usepackage{cancel}
\usepackage{amssymb,amsmath}
\usepackage{ifxetex,ifluatex}
\usepackage{fixltx2e} % provides \textsubscript
\ifnum 0\ifxetex 1\fi\ifluatex 1\fi=0 % if pdftex
  \usepackage[T1]{fontenc}
  \usepackage[utf8]{inputenc}
\else % if luatex or xelatex
  \ifxetex
    \usepackage{mathspec}
  \else
    \usepackage{fontspec}
  \fi
  \defaultfontfeatures{Ligatures=TeX,Scale=MatchLowercase}
\fi
% use upquote if available, for straight quotes in verbatim environments
\IfFileExists{upquote.sty}{\usepackage{upquote}}{}
% use microtype if available
\IfFileExists{microtype.sty}{%
\usepackage{microtype}
\UseMicrotypeSet[protrusion]{basicmath} % disable protrusion for tt fonts
}{}
\usepackage[margin=1in]{geometry}
\usepackage{hyperref}
\hypersetup{unicode=true,
            pdftitle={Correcciones de Segunda Ronda de Ejercicios},
            pdfauthor={http://synergy.vision/},
            pdfborder={0 0 0},
            breaklinks=true}
\urlstyle{same}  % don't use monospace font for urls
\usepackage{graphicx,grffile}
\makeatletter
\def\maxwidth{\ifdim\Gin@nat@width>\linewidth\linewidth\else\Gin@nat@width\fi}
\def\maxheight{\ifdim\Gin@nat@height>\textheight\textheight\else\Gin@nat@height\fi}
\makeatother
% Scale images if necessary, so that they will not overflow the page
% margins by default, and it is still possible to overwrite the defaults
% using explicit options in \includegraphics[width, height, ...]{}
\setkeys{Gin}{width=\maxwidth,height=\maxheight,keepaspectratio}
\IfFileExists{parskip.sty}{%
\usepackage{parskip}
}{% else
\setlength{\parindent}{0pt}
\setlength{\parskip}{6pt plus 2pt minus 1pt}
}
\setlength{\emergencystretch}{3em}  % prevent overfull lines
\providecommand{\tightlist}{%
  \setlength{\itemsep}{0pt}\setlength{\parskip}{0pt}}
\setcounter{secnumdepth}{0}
% Redefines (sub)paragraphs to behave more like sections
\ifx\paragraph\undefined\else
\let\oldparagraph\paragraph
\renewcommand{\paragraph}[1]{\oldparagraph{#1}\mbox{}}
\fi
\ifx\subparagraph\undefined\else
\let\oldsubparagraph\subparagraph
\renewcommand{\subparagraph}[1]{\oldsubparagraph{#1}\mbox{}}
\fi

%%% Use protect on footnotes to avoid problems with footnotes in titles
\let\rmarkdownfootnote\footnote%
\def\footnote{\protect\rmarkdownfootnote}

%%% Change title format to be more compact
\usepackage{titling}

% Create subtitle command for use in maketitle
\newcommand{\subtitle}[1]{
  \posttitle{
    \begin{center}\large#1\end{center}
    }
}

\setlength{\droptitle}{-2em}
  \title{Correcciones de Segunda Ronda de Ejercicios}
  \pretitle{\vspace{\droptitle}\centering\huge}
  \posttitle{\par}
\subtitle{José}
  \author{\url{http://synergy.vision/}}
  \preauthor{\centering\large\emph}
  \postauthor{\par}
  \date{}
  \predate{}\postdate{}


\begin{document}
\maketitle

Hacer pull a tus ejercicios.

Si las fórmulas tienen fracciones colocar antes
\texttt{\textbackslash{}displaystyle}

Las fórmulas colocarlas entre
\texttt{\$\$\ colocarm\ aqui\ la\ fórmula\ \$\$} y son varias colocarlas
entre

\texttt{\$\$}

\texttt{\textbackslash{}begin\{array\}\{\}\{\}}

\texttt{colocar\ aqui\ las\ fórmulas}

\texttt{\textbackslash{}end\{array\}}

\texttt{\$\$}

Para los decimales usar . (punto) no , (coma).

De hecho en tu ejercicio ya yo modifique varias fórmulas, revisa los
cambios.

\begin{enumerate}
\def\labelenumi{\arabic{enumi}.}
\item
\end{enumerate}

Piensalo como una distribución Geometrica.

\begin{enumerate}
\def\labelenumi{\arabic{enumi}.}
\setcounter{enumi}{1}
\item
\end{enumerate}

Sea

\[
U=\left\{
\begin{matrix}
C-15-1000 & si & \text{A  ocurre}\\
C-15 & si & \text{A no ocurre}
\end{matrix}
\right .
\]

\(P(A)=\frac{2}{100}=0.02\)

\(P(A^c)=0.98\)

Por definición del valor esperado tenemos:

\(E(U)=(C-15-1000)\quad P(A)+(C-15)\quad (A^c)\)

\begin{enumerate}
\def\labelenumi{\arabic{enumi}.}
\setcounter{enumi}{2}
\item
\end{enumerate}

\(Y=\) pérdidas de una póliza de seguros de 85.000

\[y=\left\{
\begin{matrix}
85.000 & \text{con probabilidad}& 0.001\\
42.500 & \text{con probabilidad}& 0.01
\end{matrix}
\right .
\]

\begin{enumerate}
\def\labelenumi{\arabic{enumi}.}
\setcounter{enumi}{3}
\item
\end{enumerate}

Definir la variable

\(X \quad como\quad 10^y\) y \(y={0,1,2}\)

Así \(x={0,10,20}\)

\(\mu=E(x)= \sum xP(X=x)\)

\(\sigma^2=E[(x-\mu)^2]=E(x^2)-\mu^2\)

\begin{enumerate}
\def\labelenumi{\arabic{enumi}.}
\setcounter{enumi}{4}
\item
\end{enumerate}

Como los eventos \(B_1, B_2, B_3\) son disjuntos tenemos

\(P(A)=P(A\cup B_1) + P(A\cup B_2) + P(A\cup B_3)\)

Por probabilidad condicional

\(P(A)= P(B_1)P(A_1B_1)+P(B_2)P(A_1B_2)+P(B_3)P(A_1B_3)\)

18).

\begin{enumerate}
\def\labelenumi{\arabic{enumi}.}
\item
  Cambiar \(P(x_i+y=2)\) en la segunda columna de la tabla. Colocar la
  fórmula del valor esperado condicionado.
\item
  Arreglar la tabla colocar en la primera columna retornos y arriba
  probabilidad conjunta.
\item
  Colocar las fórmulas y especificar que valores toma \(R_A\) y \(R_B\)
  de esperanza y varianza que se usaron para el cálculo.
\item
  Modificar el enunciado. La varianza de la rentabilidad de A es cercana
  a colocar fórmulas.
\item
  Colocar fórmulas y además escribir las fórmulas como lo hice en este
  ejercicio.
\item
  Colocar fórmulas y además escribirlas como las indicaciones del
  ejercicio numero 5.
\item
  Definir una variable aleatoria como DJIA aumente y colocar la fórmula.
\item
  Colocar los conjuntos entre \{\} no{[}{]} Colocar fórmulas en cada
  caso.
\item
  ¿por qué?
\item
  Qué es \(n_1,n_2,n_3\) explicar con detalles porque esas esta fórmula.
\item
  Explicar.
\end{enumerate}

19).

\begin{enumerate}
\def\labelenumi{\arabic{enumi}.}
\setcounter{enumi}{4}
\tightlist
\item
  Modificar las fórmulas \(Var(R_c)\) para que no se salga de la hoja.
\end{enumerate}


\end{document}
